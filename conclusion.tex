\chapter*{Conclusion générale}
\label{sec:conc}
\addcontentsline{toc}{chapter}{\nameref{sec:conc}}


\begin{appendices}
  
\chapter{Évolution différentielle avec Python}
  
  Le fichier contenant contenant la logique principale de la technique est \pyth{objects.py}:

  \inputminted[mathescape,
               linenos,
               numbersep=5pt,
               frame=lines,
               framesep=2mm,
               breaklines=true,
               fontsize=\scriptsize]{Python}{/home/youssef/PycharmProjects/DEPV/objects.py}
  
\chapter{Code de la stratégie métaheuristique}
  
    \inputminted[mathescape,
               linenos,
               numbersep=5pt,
               frame=lines,
               framesep=2mm,
               breaklines=true,
               fontsize=\scriptsize]{Python}{/home/youssef/PycharmProjects/DEPV/metaheuristic.py}
               
\chapter{Code de l'analyse de cohérence}
\inputminted[mathescape,
               linenos,
               numbersep=5pt,
               frame=lines,
               framesep=2mm,
               breaklines=true,
               fontsize=\scriptsize]{Python}{/home/youssef/PycharmProjects/DEPV/consistency.py}
\inputminted[mathescape,
               linenos,
               numbersep=5pt,
               frame=lines,
               framesep=2mm,
               breaklines=true,
               fontsize=\scriptsize]{Python}{/home/youssef/PycharmProjects/DEPV/metaconsistency.py}
               
\chapter{Données expérimentales}
\begin{figure}[H]
  \label{tab:apprtc}
  \begin{minipage}[t]{0.5\textwidth}
    \centering
    \captionof{table}{Cellule R.T.C France 57 mm}
    \pgfplotstabletypeset[
                          multicolumn names, % allows to have multicolumn names
                          col sep=comma, % the seperator in our .csv file
                          display columns/0/.style={
                                                    column name=$Voltage$, % name of first column
                                                    column type={S},string type},  % use siunitx for formatting
                          display columns/1/.style={
                                                    column name=$Current$,
                                                    column type={S},string type},
                          every head row/.style={
                                                    before row={\toprule}, % have a rule at top
                                                    after row={
                                                                \si{\volt} & \si{\ampere}\\ % the units seperated by &
                                                                \midrule} % rule under units
                                                                },
                          every last row/.style={after row=\bottomrule}, % rule at bottom
                          ]{/home/youssef/PycharmProjects/DEPV/data/RTC33D1000W.csv} % filename/path to file
  \end{minipage}%
  \begin{minipage}[t]{0.5\textwidth}
    \centering
    \captionof{table}{Module Schutten Solar STM6-40/36}
    \pgfplotstabletypeset[
                          multicolumn names, % allows to have multicolumn names
                          col sep=comma, % the seperator in our .csv file
                          display columns/0/.style={
                                                    column name=$Voltage$, % name of first column
                                                    column type={S},string type},  % use siunitx for formatting
                          display columns/1/.style={
                                                    column name=$Current$,
                                                    column type={S},string type},
                          every head row/.style={
                                                    before row={\toprule}, % have a rule at top
                                                    after row={
                                                                \si{\volt} & \si{\ampere}\\ % the units seperated by &
                                                                \midrule} % rule under units
                                                                },
                          every last row/.style={after row=\bottomrule}, % rule at bottom
                          ]{/home/youssef/PycharmProjects/DEPV/data/STM6_4036} % filename/path to file
  \end{minipage}
\end{figure}

\begin{table}
    \centering
    \captionof{table}{Module Photowatt-PWP 201}
      \pgfplotstabletypeset[
                            multicolumn names, % allows to have multicolumn names
                            col sep=comma, % the seperator in our .csv file
                            display columns/0/.style={
                                                      column name=$Voltage$, % name of first column
                                                      column type={S},string type},  % use siunitx for formatting
                            display columns/1/.style={
                                                      column name=$Current$,
                                                      column type={S},string type},
                            every head row/.style={
                                                      before row={\toprule}, % have a rule at top
                                                      after row={
                                                                  \si{\volt} & \si{\ampere}\\ % the units seperated by &
                                                                  \midrule} % rule under units
                                                                  },
                            every last row/.style={after row=\bottomrule}, % rule at bottom
                            ]{/home/youssef/PycharmProjects/DEPV/data/PWP} % filename/path to file
\end{table}

\end{appendices}

% \section{objects.py}
%
% \section{examples.py}
