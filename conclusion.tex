\chapter*{Conclusion générale}
\label{sec:conc}
\addcontentsline{toc}{chapter}{\nameref{sec:conc}}
Dans ce travail, nous avons étudié une approche permettant de modéliser le comportement électrique des cellules solaires photovoltaïques en estimant les paramètres des modèles d'éléments localisés à diodes. Les modèles simple et double diodes ont été présenté avec le rôle et l'influence de chaque paramètre sur la caractéristique IV du modèle. On a ensuite proposé de formuler l'extraction des paramètres en problème d'optimisation mathématique en se basant sur l'évolution différentielle qui est une technique évolutionnaire relativement récente. La fonction W de Lambert a été utilisée pour permette la reconstruction des caractéristique I-V à partir des paramètres du modèle et donc de calculer l'erreur quadratique moyenne qu'on utilise comme fonction objectif guidant la sélection des solutions de l'évolution différentielle. Nous avons aussi proposé une stratégie pour déterminer les paramètres de contrôle optimaux de l'évolution différentielle en l'utilisant à un ordre supérieur.

Un outil informatique a été développé sous le langage de programmation Python et ses bibliothèques scientifiques. Son interface programmatique a été utilisée pour appliquer cette technique sur une cellule photovoltaïque très largement étudiée dans la littérature (R.T.C France 57mm) et deux modules photovoltaïques: le monocristallin Schutten Solar STM6-40/36 et polycristallin Photowatt-PWP 201. On a pu constater que l'évolution différentielle avec la fonction W Lambert est plus précises que les autres techniques évolutionnaires similaires de l'état de l'art. En raison de la nature stochastique de cette méthode, on a dû effectuer une analyse de stabilité et de cohérence qui a montré que l'évolution différentielle converge systématiquement vers le minimum global. De plus, la stratégie métaheuristique qu'on a proposé a permit de réduire considérablement le temps de calcul en retrouvant les paramètres de contrôle optimaux pour une convergence encore plus rapide nécessitant quelques dizaines d'itérations seulement.

Les circuit électriques à diodes parviennent à modéliser les cellules photovoltaïques grâce à leurs jonctions PN sur lesquelles se base toute le principe de conversion d'énergie solaire en énergie électrique. Cependant, les technologies émergentes en photovoltaïques commencent à explorer d'autres méthodes de conversion parfois même contournant le besoin d'une jonction PN. La modélisation par circuit à diode est bien évidemment inadéquate pour ce genre de cellules, elles nécessiterons des modèles plus spécialisés, tenant en compte le mécanisme principal de séparation des charges, des propriétés opto-électroniques des matériaux utilisés et de tout un ensemble de phénomènes physiques affectant le comportement électrique de la cellule. Les travaux de recherches futurs dans ce domaine devraient évaluer la viabilité des circuits à éléments localisés pour ces nouvelles technologies. L'application des techniques évolutionnaires dépend du fait que le nombre de paramètres à optimiser soit limité, puisque l'optimisation sur des espaces de recherche avec un nombre considérable de dimensions est toujours très coûteuse en temps et en capacité de calcul, ce qui rends la convergence vers le minimum global beaucoup plus difficile et moins garantie.


\begin{appendices}
  
\chapter{Évolution différentielle avec Python}
\scriptsize   Tout le code est disponible en ligne. Pour le telecharger, il suffit d'executer \pyth{git clone https://github.com/nightowl97/DEPV.git}.
  \inputminted[mathescape,
               linenos,
               numbersep=5pt,
               frame=lines,
               framesep=2mm,
               breaklines=true,
               fontsize=\scriptsize]{Python}{/home/youssef/PycharmProjects/DEPV/objects.py}
               
\chapter{Code de l'interface graphique}
    main.py:
    \inputminted[mathescape,
               linenos,
               numbersep=5pt,
               frame=lines,
               framesep=2mm,
               breaklines=true,
               fontsize=\scriptsize]{Python}{/home/youssef/PycharmProjects/DEPV/main.py}
    main.kv:
    \inputminted[mathescape,
               linenos,
               numbersep=5pt,
               frame=lines,
               framesep=2mm,
               breaklines=true,
               fontsize=\scriptsize]{Python}{/home/youssef/PycharmProjects/DEPV/main.kv}
  
\chapter{Code de la stratégie métaheuristique}
  
    \inputminted[mathescape,
               linenos,
               numbersep=5pt,
               frame=lines,
               framesep=2mm,
               breaklines=true,
               fontsize=\scriptsize]{Python}{/home/youssef/PycharmProjects/DEPV/metaheuristic.py}
               
\chapter{Code de l'analyse de cohérence}
\inputminted[mathescape,
               linenos,
               numbersep=5pt,
               frame=lines,
               framesep=2mm,
               breaklines=true,
               fontsize=\scriptsize]{Python}{/home/youssef/PycharmProjects/DEPV/consistency.py}
\inputminted[mathescape,
               linenos,
               numbersep=5pt,
               frame=lines,
               framesep=2mm,
               breaklines=true,
               fontsize=\scriptsize]{Python}{/home/youssef/PycharmProjects/DEPV/metaconsistency.py}
               
\chapter{Données expérimentales}
\begin{figure}[H]
  \label{tab:apprtc}
  \begin{minipage}[t]{0.5\textwidth}
    \centering
    \captionof{table}{Cellule R.T.C France 57 mm}
    \pgfplotstabletypeset[
                          multicolumn names, % allows to have multicolumn names
                          col sep=comma, % the seperator in our .csv file
                          display columns/0/.style={
                                                    column name=$Voltage$, % name of first column
                                                    column type={S},string type},  % use siunitx for formatting
                          display columns/1/.style={
                                                    column name=$Current$,
                                                    column type={S},string type},
                          every head row/.style={
                                                    before row={\toprule}, % have a rule at top
                                                    after row={
                                                                \si{\volt} & \si{\ampere}\\ % the units seperated by &
                                                                \midrule} % rule under units
                                                                },
                          every last row/.style={after row=\bottomrule}, % rule at bottom
                          ]{/home/youssef/PycharmProjects/DEPV/data/RTC33D1000W.csv} % filename/path to file
  \end{minipage}%
  \begin{minipage}[t]{0.5\textwidth}
    \centering
    \captionof{table}{Module Schutten Solar STM6-40/36}
    \pgfplotstabletypeset[
                          multicolumn names, % allows to have multicolumn names
                          col sep=comma, % the seperator in our .csv file
                          display columns/0/.style={
                                                    column name=$Voltage$, % name of first column
                                                    column type={S},string type},  % use siunitx for formatting
                          display columns/1/.style={
                                                    column name=$Current$,
                                                    column type={S},string type},
                          every head row/.style={
                                                    before row={\toprule}, % have a rule at top
                                                    after row={
                                                                \si{\volt} & \si{\ampere}\\ % the units seperated by &
                                                                \midrule} % rule under units
                                                                },
                          every last row/.style={after row=\bottomrule}, % rule at bottom
                          ]{/home/youssef/PycharmProjects/DEPV/data/STM6_4036} % filename/path to file
  \end{minipage}
\end{figure}

\begin{table}
    \centering
    \captionof{table}{Module Photowatt-PWP 201}
      \pgfplotstabletypeset[
                            multicolumn names, % allows to have multicolumn names
                            col sep=comma, % the seperator in our .csv file
                            display columns/0/.style={
                                                      column name=$Voltage$, % name of first column
                                                      column type={S},string type},  % use siunitx for formatting
                            display columns/1/.style={
                                                      column name=$Current$,
                                                      column type={S},string type},
                            every head row/.style={
                                                      before row={\toprule}, % have a rule at top
                                                      after row={
                                                                  \si{\volt} & \si{\ampere}\\ % the units seperated by &
                                                                  \midrule} % rule under units
                                                                  },
                            every last row/.style={after row=\bottomrule}, % rule at bottom
                            ]{/home/youssef/PycharmProjects/DEPV/data/PWP} % filename/path to file
\end{table}

\end{appendices}

% \section{objects.py}
%
% \section{examples.py}
